\documentclass{article}

% Set page size and margins
\pagestyle{empty}
\usepackage[letterpaper,top=2cm,bottom=2cm,left=3cm,right=3cm,margin=15truemm]{geometry}

% Useful packages
\usepackage{here}
\usepackage{amsmath}
\usepackage{ascmac}
\usepackage{graphicx}
\usepackage{autobreak}
\usepackage{longtable}

% If you want to use the pdflatex environment and support Japanese, please uncomment the following
% \usepackage[whole]{bxcjkjatype}

\title{Information in PowerNetwork}
\author{by GUILDA}

\begin{document}
\maketitle{}

\section{Graph Structure of Power Networks}
\input{data/graph.tex}
 \begin{figure}[H]
 \begin{center}
 \includegraphics[width=0.9\linewidth]{data/network_graph}
 \end{center}
 \end{figure}


\section{Bus (power flow, connected component, etc...)}
\input{data/bus.tex}

\section{Branch (Admittance, connected bus, etc...)}
\input{data/brnch.tex}

\section{Component (parameter, dynamics, etc...)}
\input{data/component.tex}

\section{Local controller (parameter, dynamics, etc...)}
\input{data/controller_local.tex}

\section{Global controller (parameter, dynamics, etc...)}
\input{data/controller_global.tex}


\end{document}